\documentclass[a4paper]{article}
\usepackage{amsmath}
\usepackage{amssymb}
\usepackage{amsfonts}
\usepackage{amsthm}

\usepackage{tikz}
\usetikzlibrary{patterns, matrix}
\tikzstyle{NE-lines}=[pattern=north east lines, pattern color=black!45]

\newcommand{\fsum}[1]{\sum_{m\geq 0}m!\left(#1\right)^{m}}
\newcommand{\Sym}{\mathcal{S}}



% \pattern{ options to tikzpicture }{ |\pi| }{ pi }{ R }
% E.g., for p = ( 3241, { (0,2), (1,3), (1,4), (4,2), (4,3) } ) type:
% \pattern{}{ 4 }{ 1/3, 2/2, 3/4, 4/1 }{ 0/2, 1/3, 1/4, 4/2, 4/3 }
%
\newcommand{\pattern}[4]{
 \raisebox{0.6ex}{
 \begin{tikzpicture}[scale=0.3, baseline=(current bounding box.center), #1]
   \foreach \x/\y in {#4}
     \fill[NE-lines] (\x,\y) rectangle +(1,1);
   \draw (0.01,0.01) grid (#2+0.99,#2+0.99);
   \foreach \x/\y in {#3}
     \filldraw (\x,\y) circle (5pt);
 \end{tikzpicture}}
}

% \newtheorem{theorem}{Theorem}[section]
% \newtheorem{proposition}[theorem]{Proposition}
% \newtheorem{lemma}[theorem]{Lemma}
% \newtheorem{corollary}[theorem]{Corollary}
% \newtheorem{openproblem}[theorem]{Open Problem}
% \newtheorem*{openproblem*}{Open Problem}
\newtheorem*{conjecture}{Conjecture}

% \theoremstyle{definition}
% \newtheorem{definition}[theorem]{Definition}
% \newtheorem{remark}[theorem]{Remark}
% \newtheorem*{remark*}{Remark}
% \newtheorem{example}[theorem]{Example}
% \newtheorem*{example*}{Example}

\title{Note: Mesh patterns and a cycle restriction}
\author{Anders Claesson, Henning Ulfarsson}
\date{\today}

\begin{document}

\maketitle

\begin{abstract}
    We resolve Conjecture 3.14 in the paper ``From Hertzprung's problem to pattern-rewriting systems'' by the first author.
\end{abstract}

The first author~\cite[Conjecture 3.14]{Cl2022} makes the following conjecture.
\begin{conjecture}
    We have
    \[\sum_{n\geq 0}|\Sym_n(p)|\mskip1mu x^n
        = \fsum{\frac{x}{1+x^2}},
        \;\text{ where }\,
        p = \pattern{}{3}{1/1,2/3/,3/2}{0/2,0/3,1/2,1/3,2/0,2/1,2/2,2/3,3/1,3/2}
    \]
\end{conjecture}

Let $F(x) = \fsum{\frac{x}{1+x^2}}$. Brualdi and Deutsch~\cite{Brualdi2012} have shown
that $F(x) = (1+x^2)A(x)$ where $A(x)$ is the generating function for the number of
permutations of $[n]$ whose disjoint cycle decompositions have no adjacent
transpositions, that is, no cycles of the form $(i,i+1)$. This is A177249 in the
OEIS~\cite{OEIS}. When we map through Foata's fundamental transformation~\cite{foata} (the left-to
right maxima variation) the lack of adjacent transpositions translates to avoidance of
the mesh patterns
\[
    q_1 = \pattern{}{2}{1/2,2/1}{0/0,0/1,0/2,1/0,1/1,1/2,2/0,2/1} \quad
    q_2 = \pattern{}{3}{1/1,2/3/,3/2}{0/1,0/2,0/3,1/1,1/2,1/3,2/0,2/1,2/2,2/3,3/1,3/2} % = \pattern{}{3}{1/1,2/3/,3/2}{0/2,0/3,1/2,1/3,2/0,2/1,2/2,2/3,3/1,3/2}
\]
By the Shading Lemma~\cite{shading}, the pattern $p$ is \emph{coincident} to the pattern $q_2$ in the
conjecture, in the sense that $\Sym_n(p) = \Sym_n(q_2)$ for all $n$. Therefore,
\[
    \Sym_n(p) = \Sym_n(q_2) =  \Sym_n(q_1, q_2) \sqcup \Sym_n(q_2) \cap Co_n(q_1).
\]
The proof now follows from the fact that permutations that contain $q_1$ and avoid $q_2$
are precisely the permutations that are the direct sum of the permutation $21$ and a
permutation that avoid $q_1$.

REWORD:
We highlight that the following is still open.
Alternatively, one may
note that the compositional inverse of $x/(1+x^2)$ is $xC(x^2)$, where
$C(x)$ is the generating function for the Catalan numbers. Thus, our
conjecture is also equivalent to $[x^n]F(xC(x^2))=n!$, which could
lead to a novel decomposition of permutations.

% This conjecture is based on computing the numbers $|\Sym_n(p)|$ for
% $n\leq 14$. They are $1$, $1$, $2$, $5$, $20$, $103$, $630$, $4475$,
% $36232$, $329341$, $3320890$, $36787889$, $444125628$, $5803850515$,
% and $81625106990$. At the time of writing this sequence is not in the
% OEIS~\cite{OEIS}. If the conjecture is true then there is, however, a
% close connection with a sequence in the OEIS, namely A177249. It is
% defined by letting $a_n$ be number of permutations of $[n]$ whose
% disjoint cycle decompositions have no adjacent transpositions, that
% is, no cycles of the form $(i,i+1)$. Let
% $F(x)=\sum_{n\geq 0}|\Sym_n(p)|x^n$ be the sought series and let
% $A(x)=\sum_{n\geq 0}a_nx^n$.  Brualdi and Deutsch~\cite{Brualdi2012}
% have shown that $(1+x^2)A(x)=F(x)$.  Our conjecture is thus equivalent
% to $|\Sym_n(p)|=a_n+a_{n-2}$ for $n\geq 2$.

\bibliographystyle{plain}
\bibliography{hertz}

\end{document}
