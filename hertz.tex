\documentclass[a4paper]{article}
\usepackage{amsmath}
\usepackage{amssymb}
\usepackage{amsfonts}
\usepackage{amsthm}

\usepackage{tikz}
\usetikzlibrary{patterns, matrix}
\tikzstyle{NE-lines}=[pattern=north east lines, pattern color=black!45]

\newcommand{\fsum}[1]{\sum_{m\geq 0}m!\left(#1\right)^{m}}
\newcommand{\Sym}{\mathcal{S}}
\newcommand{\Co}{\mathcal{C}o}



% \pattern{ options to tikzpicture }{ |\pi| }{ pi }{ R }
% E.g., for p = ( 3241, { (0,2), (1,3), (1,4), (4,2), (4,3) } ) type:
% \pattern{}{ 4 }{ 1/3, 2/2, 3/4, 4/1 }{ 0/2, 1/3, 1/4, 4/2, 4/3 }
%
\newcommand{\pattern}[4]{
 \raisebox{0.6ex}{
 \begin{tikzpicture}[scale=0.25, baseline=(current bounding box.center), #1]
   \foreach \x/\y in {#4}
     \fill[NE-lines] (\x,\y) rectangle +(1,1);
   \draw (0.01,0.01) grid (#2+0.99,#2+0.99);
   \foreach \x/\y in {#3}
     \filldraw (\x,\y) circle (5pt);
 \end{tikzpicture}}
}

% \newtheorem{theorem}{Theorem}[section]
% \newtheorem{proposition}[theorem]{Proposition}
% \newtheorem{lemma}[theorem]{Lemma}
% \newtheorem{corollary}[theorem]{Corollary}
% \newtheorem{openproblem}[theorem]{Open Problem}
% \newtheorem*{openproblem*}{Open Problem}
\newtheorem*{conjecture}{Conjecture}

% \theoremstyle{definition}
% \newtheorem{definition}[theorem]{Definition}
% \newtheorem{remark}[theorem]{Remark}
% \newtheorem*{remark*}{Remark}
% \newtheorem{example}[theorem]{Example}
% \newtheorem*{example*}{Example}

\title{Note: Mesh patterns and a cycle restriction}
\author{Anders Claesson, Henning Ulfarsson}
\date{\today}

\begin{document}

\maketitle

\begin{abstract}
    We resolve Conjecture 3.14 in the paper ``From Hertzprung's problem to pattern-rewriting systems'' by the first author.
\end{abstract}

The first author~\cite[Conjecture 3.14]{Cl2022} makes the following conjecture.
\begin{conjecture}
    We have
    \[\sum_{n\geq 0}|\Sym_n(p)|\mskip1mu x^n
        = \fsum{\frac{x}{1+x^2}},
        \;\text{ where }\,
        p = \pattern{}{3}{1/1,2/3/,3/2}{0/2,0/3,1/2,1/3,2/0,2/1,2/2,2/3,3/1,3/2}
    \]
\end{conjecture}

Let $F(x) = \fsum{\frac{x}{1+x^2}}$ and let $A(x)$ be the generating function for the
number of permutations of $[n]$ whose disjoint cycle decompositions have no adjacent
transpositions, that is, no cycles of the form $(i,i+1)$. The coefficients of $A(x)$
make up sequence A177249 in the Online Encyclopedia of Integer Sequenes~\cite{OEIS}.
Brualdi and Deutsch~\cite{Brualdi2012} have shown that these two functions are related
by the equation
\[
    F(x) = (1+x^2)A(x).
\]
Consider a permutation that has no adjacent transpositions in its cycle decomposition.
When it is mapped through Foata's fundamental transformation~\cite{foata} (the left-to
right maxima variation) the lack of adjacent transpositions translates to avoidance of
the mesh patterns
\[
    q_1 = \pattern{}{2}{1/2,2/1}{0/0,0/1,0/2,1/0,1/1,1/2,2/0,2/1} \qquad
    q_2 = \pattern{}{3}{1/1,2/3/,3/2}{0/1,0/2,0/3,1/1,1/2,1/3,2/0,2/1,2/2,2/3,3/1,3/2} % = \pattern{}{3}{1/1,2/3/,3/2}{0/2,0/3,1/2,1/3,2/0,2/1,2/2,2/3,3/1,3/2}
\]
By the Shading Lemma~\cite{shading}, the pattern $p$ is \emph{coincident} to the pattern
$q_2$, in the sense that $\Sym_n(p) = \Sym_n(q_2)$ for all $n$.
Therefore,
\[
    \Sym_n(p) = \Sym_n(q_2) =  \Sym_n(q_1, q_2) \sqcup \Co_n(q_1) \cap \Sym_n(q_2),
\]
where $\Co_n(q_1)$ is the set of permutations that contain $q_1$. The proof of the
conjecture now follows from the fact that permutations that contain $q_1$ and avoid
$q_2$ are precisely the permutations that are the direct sum of the permutation $21$ and
a permutation that avoids $q_1$.

We highlight that the first author also stated an alternative version of the conjecture,
by noting that the compositional inverse of $x/(1+x^2)$ is $xC(x^2)$, where $C(x)$ is the
generating function for the Catalan numbers. This implies that the conjecture we have
proven is equivalent to $[x^n]F(xC(x^2))=n!$. Structurally this should be viewed as a
saying that any permutation can be uniquely obtained from a permutation that avoids $p$, by
replacing each of its points with a unique permutation from a family of permutations
enumerated by $xC(x^2)$. We have not been able to describe this structural decomposition.

\bibliographystyle{plain}
\bibliography{hertz}

\end{document}
